%%% Credit for base template: Tom Leinster, University of Edinburgh
% https://www.maths.ed.ac.uk/~tl/beamer/
\documentclass{beamer}

% Beamer packages
\usepackage{beamerseminar}
\usetheme{Darmstadt}

% Other packages
\usepackage{amsmath} % Math
\usepackage{pgffor} % Loops
\usepackage{subfig} % Subfigures

%%% Options
% Gets rid of the bottom-right navigation symbols
\setbeamertemplate{navigation symbols}{}

% Frame numbers
\setbeamertemplate{footline}[frame number]

% Non-shown text to appear gray
\setbeamercovered{transparent}

%%% Title page formatting
\title{Intro to Beamer - Simple Template \& Commands}
\author{EPIC RA Orientation}
\date{August 21, 2018}

% The most important line in the document
\begin{document}

% Title slide
\begin{frame}
\titlepage
\centering
\tiny{Note: Template borrowed from Tom Leinster at the University of Edinburgh, with my further edits}
\end{frame}

%2
\section{Basics}
\begin{frame}
\frametitle{Intro to Beamer}
\begin{itemize}
	\item<1> Beamer is the most popular LaTeX document class for presentations
	\vspace{0.2cm}
	\item<2> Slides in Beamer are organized as ``frames''
	\vspace{0.2cm}
	\item<3> As with itemize in regular LaTeX, use ``\textbackslash begin'' and ``\textbackslash end'', without the quotes of course, to encapsulate any object.
	\vspace{0.2cm}
	\begin{itemize}
		\item<3> e.g. To create a new slide, write ``\textbackslash begin\{frame\}'', then the contents of the frame, then ``\textbackslash end\{frame\}''. 
		\vspace{0.2cm}
		\item<3> We will call any instance ``\textbackslash begin\{\}'' and ``\textbackslash end\{\}'' an environment.
		\vspace{0.2cm}
		\item<3> Whenever you write the ``\textbackslash begin\{\}'', don't forget ``\textbackslash end\{\}''!
	\end{itemize}
\end{itemize}
\end{frame}

%3
\begin{frame}
\frametitle{Use ``\textbackslash frametitle\{\}'' to create the slide title}
Any free text that appears within the frame environment, and outside of the frame title, appears here.
\end{frame}

%4
\begin{frame}
\frametitle{Pause}
Write ``\textbackslash pause'' to insert stop animation into the slides.
\pause
\\[1em]
This text darkens after ``\textbackslash pause''
\end{frame}

%5
\begin{frame}
\frametitle{Itemize}
\begin{itemize}
	\item Itemize and enumerate all work the same way in Beamer as they do in TeX. Just remember to place the ``\textbackslash begin\{itemize\}'' and ``\textbackslash end\{itemize\}'' within the ``\textbackslash begin\{frame\}'' and ``\textbackslash end\{frame\}''.
\end{itemize}
\end{frame}

%6
\begin{frame}
\frametitle{Equations}
Same for equations. Just use ``\textbackslash begin\{align\}'' and ``\textbackslash end\{align\}'' as you otherwise would in TeX.
\begin{align*}	
	y &= \alpha + \beta x + \epsilon	.
\end{align*}	
\end{frame}

%7
\begin{frame}
\frametitle{Tables}
Tabular environment:
\begin{center}
  \begin{tabular}{ c | c  c  c }
    $\ast$ & e & a & b \\ \hline
    e & e & a & b \\ 
    a & a & b & e \\ 
    b & b & e & a
  \end{tabular}
\end{center} 
Array environment:
\[ 
\begin{array}{c | c c c}
    \ast & e & a & b \\ \hline
    e & e & a & b \\ 
    a & a & b & e \\ 
    b & b & e & a
 \end{array} 
\] 
The distinction is the latter's use of a math environment. Also, the array allows brackets or parentheses to surround the table.
\end{frame}

%8
\begin{frame}
\frametitle{Reading in a table from another file}
Use ``\textbackslash input\{path-to-file\}''. 
\\[1em]
I recommend this over hard-coded tables in the presentation, to allow for modularity when iterating through analyses. Set up your Stata/R/Python code to write to a TeX file with pre-specified formatting.
\begin{table}
	\centering
	\scriptsize{
	\begin{tabular}{lcccc}
\hline
                           & OLS      & FE      & \begin{tabular}[c]{@{}c@{}}OLS with \\ industry$\times$year \\ interactions\end{tabular} & \begin{tabular}[c]{@{}c@{}}FE with \\ industry$\times$year \\ interactions\end{tabular} \\
\hline
$\log$(COD fees per KG) & $-.0606$ & $-.0822^{**}$ & $-.0798^{**}$ & $-.0829^{**}$ \\
& (.04) & (.036) & (.038) & (.036) \\
                           \hline
Year fixed effects         & Yes      & Yes     & Yes                                                                               & Yes                                                                              \\
Province fixed effects     & Yes      & No        & Yes                                                                               &         No                                                                         \\
Firm fixed effects         &  No        & Yes     &     No                                                                              & Yes                                                                              \\
Industry$\times$year interactions &  No        &   No      & Yes                                                                               & Yes                                                                              \\ \hline
$ R^2$ & 0.05 & 0.85 & 0.20 & 0.85 \\
Observations & 291,821 & 291,821 & 291,821 & 291,821 \\
\hline          
\end{tabular}
	}
\end{table}
\end{frame}


%9
\begin{frame}
\frametitle{Figures \& Graphics}
Simply ``\textbackslash includegraphics\{path-to-file\}''. 
\\[1em]
We recommend using ``\textbackslash centering'' within a figure framework for better formatting.
\vspace{1cm}
\begin{figure}
	\includegraphics[width=0.8\linewidth]{epic_logo}
\end{figure}
\end{frame}

%10
\begin{frame}
\frametitle{Figures \& Graphics}
Side-by-side figures:
\begin{figure}[h]%
    \centering
    \subfloat{{\includegraphics[scale=0.33]{epic_logo} }}%
    \quad
    \subfloat{{\includegraphics[scale=0.33]{epic_logo} }}%
    %\caption{2 Figures side by side}%
    %\label{fig:example}%
\end{figure}
\end{frame}

%11
\begin{frame}
\frametitle{Figures \& Graphics}
Stacked side-by-side figures with a figure caption:
\begin{figure}[h]%
    \centering
    \subfloat{{\includegraphics[scale=0.33]{epic_logo} }}%
    \quad
    \subfloat{{\includegraphics[scale=0.33]{epic_logo} }}%
    \\[1em]
    \subfloat{{\includegraphics[scale=0.33]{epic_logo} }}%
    \quad
    \subfloat{{\includegraphics[scale=0.33]{epic_logo} }}%
    \vspace{0.2cm}
    \caption{Here are four EPIC logos}
\end{figure}
\end{frame}

%12
\section{More Complicated Features}
\begin{frame}
\frametitle{Uncovering an Equation in a Derivation}
Use ``\textbackslash uncover\textless1-\textgreater'' within the align environment, replacing the `1' with any natural number to order the uncovering.
\begin{align*}
	\uncover<1->{\zeta &= P\left(q_{\frac{\alpha}{2}}SE\left(\bar{x}\right) < \bar{x} - \mu < q_{1 - \frac{\alpha}{2}}SE\left(\bar{x}\right)\right)} ,
	\\[1em]
	\uncover<2->{ &= P\left(-q_{\frac{\alpha}{2}}SE\left(\bar{x}\right) > \mu - \bar{x} > -q_{1 - \frac{\alpha}{2}}SE\left(\bar{x}\right)\right),}
	\\[1em]
	\uncover<2->{&= P\left(\bar{x} -q_{\frac{\alpha}{2}}SE\left(\bar{x}\right) > \mu  > \bar{x} -q_{1 - \frac{\alpha}{2}}SE\left(\bar{x}\right)\right).}	
\end{align*}
\uncover<3>{If you wish, add [1em] to the \textbackslash\textbackslash \, line breaks for more space between the lines.}
\end{frame}

%13 - 16
\foreach \ot in {mayor, ps}	 {%% First loop over mayor vs. party secretary, i.e. official type
    \foreach \ct in {crpt, noncrpt} {% % Then whether the official was investigated for corruption or not
        % Defines a string to correspond with our "corrupt" indicator
        \ifnum\pdfstrcmp{\ct}{crpt}=0 % String comparison, i.e. "if equal". \ifnum needs a boolean as input.
            \newcommand{\ctstr}{corrupt}
        \else
            \newcommand{\ctstr}{non-corrupt}
        \fi % Don't forget this!
        
        \ifnum\pdfstrcmp{\ot}{mayor}=0 
            \newcommand{\otstr}{party secretaries}
        \else
            \newcommand{\otstr}{mayors}
        \fi % Don't forget this!
        
        \begin{frame}
	\frametitle{Figures in a loop}
            % FIGURE is here
            \begin{figure}[H]
            \caption{\textbf{Censored distribution of land sales' price for \ctstr\hspace{0.05em} \otstr)}}
            \centering
            \includegraphics[width=0.65\linewidth]{china_figures/histograms/unitprice_small_\ct_\ot_mf}
            \label{unitprice_small_\ct_\ot_mf} % Label the figure for future referencing
            \end{figure}
         \end{frame}
    }%
}%

%17
\begin{frame}
\frametitle{Any questions?}
\end{frame}

% The other most important line in the document
\end{document}
