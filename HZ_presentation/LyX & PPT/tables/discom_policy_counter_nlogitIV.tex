\begin{table}[!ht]
	\centering
		\caption{Distribution company policy reform counterfactuals: market shares and WTP}
		\scalebox{0.8}{
\begin{tabular}{lccccccc}
		\toprule
		& \multicolumn{5}{c}{Market shares} & & \\
		\cline{2-6}
               &      Grid&    Diesel& Own solar&       HPS&      None&\shortstack{Annual WTP\\per HH (INR)}&\shortstack{Annual Losses\\per HH (INR)}\\
		\midrule
		         Actual&      0.41&      0.05&      0.07&      0.05&      0.42&        &    861.45\\
\addlinespace
 \multicolumn{8}{l}{\textit{i. Expand availability and supply}} \\
		Grid everywhere&      0.69&      0.02&      0.03&      0.03&      0.23&     89.38&   1436.57\\
		
		   Extra 1 Hour&      0.45&      0.05&      0.08&      0.06&      0.36&    414.97&   1042.04\\
		  Extra 2 Hours&      0.51&      0.04&      0.07&      0.06&      0.32&   1041.80&   1287.86\\
		  \addlinespace
\multicolumn{8}{l}{\textit{ii. Remove theft}} \\
		   Grid INR 140&      0.29&      0.06&      0.10&      0.08&      0.47&   -535.93&    321.74\\
		   \addlinespace
\multicolumn{8}{l}{\textit{iii. Budget neutral reduction in theft and 5 peak hours}} \\
		   Grid INR 120&      0.45&      0.05&      0.08&      0.06&      0.36&    381.13&    843.04\\
		\bottomrule
		\multicolumn{8}{p{\textwidth}}{\footnotesize We model removing theft by raising the grid price to reported survey bill values, assuming payment rate is 100\%. In the budget neutral theft reduction with increased supply hours, we set peak hours equal to its maximum of 5 and raise the grid price until annual losses are equivalent to actual annual losses.}\\
	\end{tabular}}
\end{table}
